
% Document settings
\documentclass[11pt]{article}
\usepackage{amsmath}
\usepackage{stmaryrd}
\usepackage{bm}
\usepackage[margin=1in]{geometry}
\usepackage[pdftex]{graphicx}
\usepackage{color}
\usepackage{listings}
\usepackage{hyperref}
\usepackage{multirow}
\usepackage{multicol}
\usepackage{setspace}
\usepackage{xspace}
\usepackage{listings}

\pagestyle{plain}
\setlength\parindent{0pt}
\setlength\parskip{4pt}
\setlength{\columnseprule}{0.2pt}

% Define the speculative C++1z language.
\lstdefinelanguage{C++1z}{alsolanguage=C++,
                          escapechar=@,
                          breakatwhitespace=true,
                          breaklines=true
                          morekeywords={alignof, 
                                        decltype, 
                                        concept, 
                                        requires}}

% Default formatting for listings.
\lstset{language=C++1z,
        basicstyle=\ttfamily,
        keywordstyle=\bfseries,
        stringstyle=,
        xleftmargin=1em,
        showstringspaces=false,
        commentstyle=\rmfamily\itshape,
        columns=flexible,
        keepspaces=true,
        texcl=true}

\newcommand{\code}[1]{\lstinline @#1@}
\lstnewenvironment{codeblock}{\lstset{language=C++1z}}{}

\newcommand{\mathsc}[1]{\bm{\mathsf{#1}}}
% \newcommand{\mathsc}[1]{\mathsf{#1}}

% Types
\newcommand{\typebool}{\ensuremath{\mathsc{bool}}\xspace}
\newcommand{\typeint}{\ensuremath{\mathsc{int}}\xspace}
\newcommand{\typefloat}{\ensuremath{\mathsc{float}}\xspace}
\newcommand{\typeref}[1]{\ensuremath{\mathsc{ref}~#1}\xspace}
\newcommand{\typefn}[2]{\ensuremath{(#1)\rightarrow #2}\xspace}

% Expressions
\newcommand{\exprtrue}{\mathsc{true}}
\newcommand{\exprfalse}{\mathsc{false}}
\newcommand{\exprand}[2]{#1 \mathbin{\mathsc{and}} #2}
\newcommand{\expror}[2]{#1 \mathbin{\mathsc{or}} #2}
\newcommand{\exprxor}[2]{#1 \mathbin{\mathsc{xor}} #2}
\newcommand{\exprnot}[1]{\mathsc{not}\;#1}
\newcommand{\exprif}[3]{\mathsc{if}\;#1\;\mathsc{then}\;#2\;\mathsc{else}\;#3}
\newcommand{\exprandthen}[2]{#1 \mathbin{\mathsc{and\;then}} #2}
\newcommand{\exprorelse}[2]{#1 \mathbin{\mathsc{or\;else}} #2}
\newcommand{\expreq}[2]{#1 = #2}
\newcommand{\exprne}[2]{#1 \neq #2}
\newcommand{\exprlt}[2]{#1 < #2}
\newcommand{\exprgt}[2]{#1 > #2}
\newcommand{\exprle}[2]{#1 \leq #2}
\newcommand{\exprge}[2]{#1 \geq #2}
\newcommand{\expradd}[2]{#1 + #2}
\newcommand{\exprsub}[2]{#1 - #2}
\newcommand{\exprmul}[2]{#1 * #2}
\newcommand{\exprquo}[2]{#1 \mathbin{\mathsc{div}} #2}
\newcommand{\exprrem}[2]{#1 \mathbin{\mathsc{rem}} #2}
\newcommand{\exprneg}[1]{-#1}
\newcommand{\exprrec}[1]{/#1}

% Statements
\newcommand{\stmtbreak}{\ensuremath{\mathsc{break}}\xspace}
\newcommand{\stmtcont}{\ensuremath{\mathsc{continue}}\xspace}
\newcommand{\stmtblock}[1]{\ensuremath{\{#1\}}\xspace}
\newcommand{\stmtwhile}[2]{\ensuremath{%
  \mathsc{while}~#1~\mathsc{do}~#2%
  }\xspace}
\newcommand{\stmtif}[3]{\ensuremath{%
  \mathsc{if}~#1~\mathsc{then}~#2~\mathsc{else}~#3%
  }\xspace}
\newcommand{\stmtret}[1]{\ensuremath{%
  \mathsc{return}~#1%
  }\xspace}
\newcommand{\stmtexpr}[1]{\ensuremath{#1}\xspace}
\newcommand{\stmtdecl}[1]{\ensuremath{#1}\xspace}

\newcommand{\decllet}[4]{\ensuremath{%
  \mathsc{#1}~#2\mathbin{:}#3\mathbin{=}#4%
  }\xspace}
\newcommand{\declvar}[3]{\decllet{var}{#1}{#2}{#3}}
\newcommand{\declref}[3]{\decllet{ref}{#1}{#2}{#3}}
\newcommand{\declfn}[4]{\ensuremath{%
  \mathsc{fun}~#1(#2) \rightarrow #3~#4%
  }\xspace}

\title{MiniC Language Specification}

\begin{document}
\maketitle

This document describes the abstract and concrete syntax and semantics of a small C-like programming language (albeit with a very different syntax for declarations).

\section{Basics}

The fundamental storage unit is the \emph{byte}, which is a sequence of 8 bits.
The storage (or memory) available to a program consists of one or more sequences of contiguous bytes. 
Every byte has a unique address.

The following paragraphs describe the entities of the language. An \emph{entity} is an abstract element of a program that is described by, created by, or
referred to in a program.

A \emph{value} is element of a set.
For example the value zero is an element in the set of integers and the set of real numbers.
Similarly, the value true is an element in the set boolean values; the other value is false.
A value can also be an address.
When this document refers to specific values, they are spelled in English using the normal font.

A \emph{type} describes objects, references, functions, and expressions.
The meaning of the type is determined by the the entity it describes.

An \emph{object} is region of storage, comprised of a fixed length sequence
of bytes.
Note that every object has an address.
Every object has a type, which determines the size of the object and how its stored bits can be interpreted as a value (i.e., an object stores a value).
Objects are created by definitions or when a temporary object is created.

A \emph{reference} can be thought of as an alias (name) for an object or alternatively as an address for the object.
Every reference has a type, which determines the set of objects to which the reference can bind.
It is unspecified whether a reference requires storage.
References are created by definitions.

A \emph{function} maps a sequence of input values, called \emph{arguments},
to an output value, called its \emph{return value}.
Every function has a type, which determines the arguments that it can accept and as inputs the type of value that it returns as an output.
Functions are created by definitions.

\section{Abstract syntax and semantics}

This section describes the \emph{abstract syntax} of the language, which is comprised of four sub-languages: types ($t$), expressions ($e$), statements ($s$), and definitions $(d)$.
Each sub-language is defined (recursively) as a set of strings belonging to that set.
Note that these are ``strings'' in the abstract sense; they will be represented as abstract syntax trees.

The meaning of each string is determined by the semantic specifications in this document. 
The semantic specification determines several properties of the language:
\begin{itemize}
\item the static properties associated with a string (or \emph{static semantics});
\item the dynamic properties associated with a string (or \emph{dynamic semantics});
\end{itemize}
The static and dynamic semantics of strings constrain the set of strings that could otherwise be formed by simply enumerating valid combinations.
For example, the expression \code{1 / n} is valid if and only if \code{n} refers to an object of type \code{int} (statically) and the value of that object is never 0 (dynamically).

A \emph{program} is a sequence of definitions (substrings of $d$); that is,
it is also a string.

\subsection{Types}

The language has the following kinds of types:

\[
\begin{array}{lcll}
t &::=& \typebool  & \text{boolean values} \\
  &   & \typeint   & \text{integer values} \\
  &   & \typefloat & \text{floating point values} \\
  &   & \typeref{t_1}   & \text{references to objects} \\
  &   & \typefn{t_1, t_2, \ldots, t_n}{t_r}
                  & \text{functions}
\end{array}
\]

% The type \typebool describes the values true and false.

% The type \typeint describes integer values in the left-open range $[-2^{32-1}, 2^{32-1})$.

% The type \typefloat describes single precision IEEE 754 floating point values.

% The types \typebool, \typeint, and \typefloat are collectively called the \emph{object types}, meaning they can be used to define objects.

% The reference types \typeref{t} describes references to objects. 
% These value can be represented as the address of the object in memory. 

% The set of function types \typefn{t_1, t_2, \ldots, t_n}{t_r} describes
% functions taking $n$ arguments, whose corresponding types are $t_1$, $t_2$, $\ldots$, and $t_n$, and returning a value of type $t_r$.
% These values can be represented as the address of the function in memory.

\subsection{Expressions}

The language has the following kinds of expressions:

\begin{multicols}{2}
\[
\begin{array}{lcll}
e &::=& \exprtrue & \\
  &   & \exprfalse & \\
  &   & n & \text{integer literals}\\
  &   & x & \text{identifiers}\\
  &   & \exprand{e_1}{e_2} & \text{logical and}\\
  &   & \expror{e_1}{e_2} &\text{logical or}\\
  &   & \exprnot{e_1} & \text{logical negation}\\
  &   & \exprif{e_1}{e_2}{e_3} & \text{conditional}\\
  &   & \expreq{e_1}{e_2} & \text{equal to}\\
  &   & \exprne{e_1}{e_2} & \text{not equal to}\\
  % &   & \ldots\\
\end{array}
\]
\vfill
\columnbreak
\[
\begin{array}{lll}
        % \ldots\\
        \exprlt{e_1}{e_2} & \text{less than}\\
        \exprgt{e_1}{e_2} & \text{greater than}\\
        \exprle{e_1}{e_2} & \text{less than or equal to}\\
        \exprge{e_1}{e_2} & \text{greater than or equal to}\\
        \expradd{e_1}{e_2} & \text{addition}\\
        \exprsub{e_1}{e_2} & \text{subtraction}\\
        \exprmul{e_1}{e_2} & \text{multiplication}\\
        \exprquo{e_1}{e_2} & \text{quotient of division}\\
        \exprrem{e_1}{e_2} & \text{remainder of division}\\
        \exprneg{e_1} & \text{negation}\\
        \exprrec{e_1} & \text{reciprocal}\\
\end{array}
\]
\end{multicols}


\subsection{Statements}

The language has the following kinds of statements:

\[
\begin{array}{lcll}
s &::=& \stmtbreak              & \text{break statements} \\
  &   & \stmtcont               & \text{continue statement} \\
  &   & \stmtblock{s_1, s_2, \ldots, s_n}
                                & \text{compound statements} \\
  &   & \stmtwhile{e_1}{s_1}    & \text{while statements} \\
  &   & \stmtif{e}{s_1}{s_2}    & \text{if statements} \\
  &   & \stmtret{e}             & \text{return statements} \\
  &   & \stmtexpr{e}            & \text{expressions} \\
  &   & \stmtdecl{d}            & \text{local definitions} \\
\end{array}
\]

\subsection{Declarations}

The language has the following kinds of declarations:

\[
\begin{array}{lcll}
d &::=& \declvar{n}{t}{e}       & \text{object definitions} \\
  &   & \declref{n}{t}{e}       & \text{reference definitions} \\
  &   & \declfn{n}{d_1, d_2, \ldots, d_n}{t}{s}
                                & \text{function definitions} \\
\end{array}
\]



% An expression is a sequence of operands and operators that specifies a 
% computation. The evaluation of an expression results in a value. The type of an 
% expression determines how expressions can be combined to produce complex 
% computations and the kind of value it produces. The following paragraphs define
% the requirements on operands and the result types of each expression as well 
% the values they produce.

% The order in which an expression's operands are evaluated is unspecified unless 
% otherwise noted.

% The expressions $\exprtrue$ and $\exprfalse$ have type $\typebool$ and the
% values $\top$ and $\bot$, respectively.

% Integer literals have type $\typeint$. The value of an integer literal is the
% one indicated by the expression.

% The operands of the expressions $\exprand{e_1}{e_2}$, $\expror{e_1}{e_2}$,
% and $\exprxor{e_1}{e_2}$ shall have type $\typebool$. The result type of each
% is $\typebool$. 
% The result of $\exprand{e_1}{e_2}$ is $\top$ if both operands are $\top$ 
% and $\bot$ otherwise.
% The result of $\expror{e_1}{e_2}$ is $\bot$ if both operands are $\bot$ 
% and $\top$ otherwise.
% The result of $\exprxor{e_1}{e_2}$ is $\top$ if the operands are different 
% and $\bot$ otherwise.

% The operand of $\exprnot{e_1}$ shall have type $\typebool$, and the type of the
% expression is $\typebool$. The result of the expression is $\top$ when
% the $e_1$ is $\bot$ and $\bot$ otherwise.
% Note that $\exprnot{e_1}$ is equivalent to $\exprxor{e_1}{1}$.

% In the expression $\exprif{e_1}{e_2}{e_3}$, the type of $e_1$ shall be 
% $\typebool$, and $e_2$ and $e_3$ shall have the same type. The type of the
% expression is the type of $e_2$ and $e_3$. The result of expression is
% determined by first evaluating $e_1$. If that value is $\top$ then the result
% of the expression is the value of $e_2$, otherwise, it is the value of $e_3$.
% Only one of $e_2$ or $e_3$ is evaluated.

% The operands of the expressions $\expreq{e_1}{e_2}$ and $\exprne{e_1}{e_2}$ 
% shall have the same type. The result type is $\typebool$.
% The result of $\expreq{e_1}{e_2}$ is $\top$ if $e_1$ and $e_2$ are equal 
% and $\bot$ otherwise. 
% The result of $\exprne{e_1}{e_2}$ is $\top$ if $e_1$ and $e_2$ are different 
% and $\bot$ otherwise. 
% Note that when $e_1$ and $e_2$ have type $\typebool$, $\exprne{e_1}{e_2}$ is
% equivalent to $\exprxor{e_1}{e_2}$.

% The operands of the expressions $\exprlt{e_1}{e_2}$, $\exprgt{e_1}{e_2}$,
% $\exprle{e_1}{e_2}$, and $\exprge{e_1}{e_2}$ shall have type $\typeint$. The 
% result type is $\typebool$.
% The result of $\exprlt{e_1}{e_2}$ is $\top$ if $e_1$ is less than $e_2$ and 
% $\bot$ otherwise. 
% The result of $\exprgt{e_1}{e_2}$ is $\top$ if $e_1$ is greater than $e_2$ and 
% $\bot$ otherwise. 
% The result of $\exprle{e_1}{e_2}$ is $\top$ if $e_1$ is less than or equal to
% $e_2$ and $\bot$ otherwise.
% The result of $\exprge{e_1}{e_2}$ is $\top$ if $e_1$ is greater than or equal
% to $e_2$ and $\bot$ otherwise.

% The operands of the expressions $\expradd{e_1}{e_2}$, $\exprsub{e_1}{e_2}$,
% $\exprmul{e_1}{e_2}$, $\exprquo{e_1}{e_2}$, and $\exprrem{e_1}{e_2}$ shall have 
% type $\typeint$. The result type is $\typeint$.
% The result of $\expradd{e_1}{e_2}$ is the sum of the operands. If the sum 
% is greater than the maximum value of $\typeint$, the result is undefined.
% The result of $\expradd{e_1}{e_2}$ is the difference resulting from the 
% subtraction of the $e_2$ from $e_1$. If the difference is less than the minimum
% value of $\typeint$, the result is undefined.
% The result of $\exprmul{e_1}{e_2}$ is the product of the operands. If the 
% product is greater than the maximum value of $\typeint$, the result is 
% undefined.
% The results of $\exprquo{e_1}{e_2}$ and $\exprrem{e_1}{e_2}$ are the quotient 
% and remainder of dividing $e_1$ by $e_2$, respectively. In either case, if
% $e_2$ is 0, the result is undefined. If $e_2$ is the minimum value of $\typeint$,
% the result is undefined. For division, the fractional part of the value is
% discarded (the value is truncated towards zero). If the expression 
% $\exprquo{a}{b}$ is defined, 
% $\expradd{\exprmul{(\exprquo{a}{b})}{b}}{\exprrem{a}{b}}$ is equal to $a$.

% The operand of $\exprneg{e_1}$ shall have type $\typeint$, and the type of the
% expression is $\typeint$. 
% The result of the expression is the additive inverse of the value of $e_1$.
% Note that $\exprneg{e_1}$ is equivalent to $\exprsub{0}{e_1}$.



% \section{Requirements}

% Implement A C++ framework (collection of classes and functions) that supports this
% language definition. This must include:

% \begin{itemize}
% \item Implement a set of classes that define the AST for the language. Prefer
% an object-oriented class hierarchy for the implementation.

% \item A facility that verifies that an expression is well-type according to
% the typing rules above.

% \item A facility that supports the creation of well-typed AST nodes. This should
% verify the types of operands before constructing the requested node.

% \item A facility that evaluates an expression (as an AST node).
% \end{itemize}

% The type checking and evaluation functions can be implemented as virtual
% functions or via the visitor design pattern. The tree creation facility does
% not require virtual functions.

% Store your work in an online repository. I recommend GitHub, GitLab, or 
% BitBucket.

% Write a short paper to serve as an overview and guide for your implementation.
% Describe the components of your project in a way that would help a newcomer
% understand the project's organization. Be sure to include a link to your online
% source code in your submission.

% \textbf{Submission:} Submit your printed homework on the due date. Send me an
% email with a link to your online source code.

% \textbf{Above and beyond}: Make sure that the classes in your hierarchy are
% well constructed (constructors, access restrictions, appropriate documentation).
% Note that the semantics of the language could be written 

% Consider adding two extra expressions:
% \[
% \begin{array}{lcl}
% e & ::= & \cdots \\
%   &     & \exprandthen{e_1}{e_2} \\
%   &     & \exprorelse{e_1}{e_2} \\
% \end{array}
% \]

% The expression $\exprandthen{e_1}{e_2}$ is equivalent to 
% $\exprif{e_1}{e_2}{\exprfalse}$.
% The expression $\exprorelse{e_1}{e_2}$ is equivalent to
% $\exprif{\exprtrue}{e_2}{e_2}$.

\end{document}



